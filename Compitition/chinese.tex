\documentclass[12pt, a4paper]{article}

% =============================================
% 宏包引入
% =============================================
\usepackage[UTF8]{ctex}     % 中文支持
\usepackage{geometry}       % 页面布局
    \geometry{left=2.5cm, right=2.5cm, top=2.5cm, bottom=2.5cm}
\usepackage{amsmath, amssymb} % 数学公式
\usepackage{amsthm}         % 定理环境
\usepackage{graphicx}       % 图片插入
\usepackage{float}          % 图片浮动控制
\usepackage{booktabs}       % 三线表
\usepackage{array}          % 表格样式
\usepackage{hyperref}       % 超链接
\usepackage{fancyhdr}       % 页眉页脚
\usepackage{listings}       % 代码展示
\usepackage{xcolor}         % 颜色
\usepackage{colortbl}        % 表格颜色

% =============================================
% 格式设置
% =============================================
% 页眉页脚
\pagestyle{fancy}
\fancyhf{}
\lhead{A题:烟幕干扰弹的投放策略}
\rhead{数学建模论文}
\cfoot{\thepage}

% 代码块样式
\lstset{
    basicstyle=\small\ttfamily,
    numbers=left,
    keywordstyle=\color{blue},
    commentstyle=\color{green!50!black},
    frame=shadowbox,
    breaklines=true
}

% 定理环境定义
\newtheorem{theorem}{定理}[section]
\newtheorem{definition}[theorem]{定义}

% 标题设置
\title{\textbf{烟幕干扰弹的无人机投放策略研究}}
\author{队伍编号:\underline{\hspace{3cm}}}
\date{\today}

% =============================================
% 正文开始
% =============================================
\begin{document}

% --- 标题页 ---
\maketitle
\thispagestyle{empty} % 首页不显示页码

% --- 摘要 ---
\begin{abstract}
\normalsize
定点精确投放的烟幕干扰弹,可以在目标前方特定空域形成遮蔽,有效地干扰敌方导弹。
本文针对无人机投放烟幕干扰弹以掩护地面目标的问题,建立了基于空气动力学和空间几何的数学模型。
通过分析导弹与烟幕云团的相对运动关系,获得了不同情境下的最优投放策略,以最大化有效遮蔽时间。

针对\textbf{问题一},我们建立了烟幕弹自由落体与扩散模型。已知无人机 $FY1$ 的飞行参数,通过计算烟幕弹起爆时的空间位置与扩散半径,结合导弹 $M1$ 的运动轨迹,求解出两者在时空上的交集,从而得出有效遮蔽时长。

针对\textbf{问题二},这是一个单机单弹对单导弹的优化问题。以无人机飞行方向、速度、投放点和起爆点为决策变量,以遮蔽时间为目标函数,建立了非线性规划模型。利用遗传算法(Genetic Algorithm)进行求解,确定了使遮蔽时间最长的飞行参数。

针对\textbf{问题三},在问题二的基础上引入多枚烟幕弹。考虑到烟幕生成的不连续性,建立了多波次干扰模型。通过规划三枚烟幕弹的时间间隔与空间分布,确立了“接力式”遮蔽策略,并将结果存入 \texttt{result1.xlsx}。

针对\textbf{问题四}和\textbf{问题五},模型扩展为多机多弹协同对抗多枚导弹(M1, M2, M3)。我们将问题转化为多智能体协同任务分配(Task Assignment)与轨迹规划问题。构建了以总遮蔽效益最大化为目标的整数规划模型,给出了各架无人机的具体任务分配与投放指令。

\vspace{1em}
\noindent \textbf{关键词:} 烟幕干扰;动态几何遮蔽;多目标优化;投放策略;多机协同
\end{abstract}

\newpage
\tableofcontents % 生成目录
\newpage
\setcounter{page}{1} % 正文从第1页开始

% =============================================
% 1. 问题重述
% =============================================
\section{问题重述}

\subsection{问题背景}
现代战争中,利用无人机投放烟幕干扰弹是保护地面重要目标的有效手段,具有成本低、效费比高等优点。烟幕弹在空中起爆后形成云团,形成遮蔽,干扰敌方导弹。
本题要求在给定的战场环境下(包含来袭导弹位置、无人机位置、目标位置及物理参数),综合考虑多个物体的运动轨迹,设计无人机的飞行与投放策略,以实现最大化的遮蔽效果。

\subsection{问题提出}
\begin{itemize}
    \item \textbf{问题 1}:对于给定单无人机 $FY1$ 的具体飞行参数,根据运动学建立数学模型
    计算得出导弹 $M1$ 的有效遮蔽时长。
    \item \textbf{问题 2}:从问题一的特殊情况出发拓展到了一般情况,
    优化单无人机 $FY1$ 投放 1 枚烟幕弹的策略(方向、速度、投放点、起爆点),使遮蔽时间最大。
    \item \textbf{问题 3}:面对单无人机 $FY1$ 投放多枚烟幕弹的情况,规划烟雾弹的投递时序,实现总遮蔽时间的最大化。
    \item \textbf{问题 4}:不同于单机多枚烟雾弹的情形,情况变成了多机协同($FY1-FY3$)各投 1 枚弹对抗 $M1$ ,需要找到其投放的最优策略。
    \item \textbf{问题 5}:全要素协同场景,需要找到(5架无人机,每架最多3枚弹)对抗 3 枚导弹($M1-M3$)的最优策略。
\end{itemize}
% =============================================
% 2. 问题分析
% =============================================
\section{问题分析}
\subsection{问题一的分析}
问题一是最简单的情景,要求计算得到给定参数下的有效遮蔽时间。
首先以烟幕弹爆炸瞬间的空间位置与导弹的空间位置作为初始值,后续考虑扩散半径,结合导弹运动轨迹,求解两者的交点,从而得到有效遮蔽时长。
\subsection{问题二的分析}
问题二是问题一的推广,需要将无人机飞行方向、速度、投放点、起爆点作为决策变量进行优化,以最大化遮蔽时间作为目标函数,获取在满足一定的约束条件下的最佳解。
实现烟幕弹的最佳投放策略。
\subsection{问题三的分析}
问题三是问题二的优化,需要控制好烟幕弹投放的时序,完成烟幕弹的协同投放,以实现多弹接力式的遮蔽效果,从而实现最大化有效遮蔽时间。
\subsection{问题四的分析}
问题四是多机协同的情况,需要考虑多架无人机在时空上的协同,实现在空间与时间上的遮蔽效果的最大化。
\subsection{问题五的分析}
问题五是最复杂的情景,涉及多机多弹对抗多枚导弹的情况。需要综合考虑无人机的任务分配、轨迹规划以及烟幕弹的投放时序等多个因素,建立一个综合性的优化模型,以实现整体遮蔽效益的最大化。
但是直接考虑所有的决策变量,时间复杂度与空间复杂度较高,无法直接求解。
因而需要对问题进行简化,将问题转化为多智能体协同任务分配与投放策略问题。



% =============================================
% 3. 模型假设
% =============================================
\section{模型假设}
为简化问题,主要假设如下:
\begin{enumerate}
    \item 假设烟幕弹在投放后忽略空气阻力,仅受重力作用做平抛运动。
    \item 假设烟幕云团形成球状且半径在有效时间内保持稳定(或按题目给定速度扩散/下沉)。
    \item 假设来袭导弹做匀速直线运动,不进行机动变轨。
    \item 假设雷达发现目标时刻为 $t=0$。
    \item 忽略无人机调整飞行方向时的转弯半径,视为瞬时变向。
    \item 假设无人机、导弹、烟幕弹、烟幕云团均为质点。
\end{enumerate}

% =============================================
% 4. 符号说明
% =============================================
\section{符号说明}
本文主要使用的符号定义如下表所示:

\begin{table}[H]
    \centering
    \caption{符号说明}
    \begin{tabular}{clc}
        \toprule[1.5pt]
        \textbf{符号} \quad    & \textbf{含义} & \textbf{单位} \\
        \midrule[1pt]
        $M_j$ \quad  & 第 $j$ 枚来袭导弹 ($j=1,2,3$) & - \\
        $FY_i$  \quad & 第 $i$ 架无人机 ($i=1,2,3,4,5$) & - \\
        $V_m$ \quad  & 导弹飞行速度 (固定 300) & m/s \\
        $V_{uav}$ \quad  & 无人机飞行速度 & m/s \\
        $(x, y, z)$ \quad  & 空间坐标位置 & m \\
        $T_{mask}$\quad   & 有效遮蔽时间 & s \\
        $t_{drop}$\quad   & 烟幕弹投放时刻 & s \\
        $t_{boom}$\quad  & 烟幕弹起爆时刻 & s \\
        $R_{cloud}$ \quad  & 烟幕云团有效遮蔽半径 & m \\
        \bottomrule[1.5pt]
    \end{tabular}
\end{table}

% =============================================
% 5. 模型建立
% =============================================
\section{模型建立}

根据题目描述首先建立统一的三维直角坐标系$Oxyz$。以假目标中心为原点$x$ 轴与 $y$ 轴位于水平面内,
$z$ 轴竖直向上,真目标位于点$P_T=\left( 0,200,h_T \right) $其中 $h_T$ 为真目标中心高度。\\
\\第$j$ 枚导弹的初始位置位于点$$P_{M_j}(0) $$
\\第$i$ 架无人机的初始位置位于点$$P_{FY_i}(0)$$
\\我们可以利用matlab绘制初始时刻的宏观战场态势图,如图\ref{fig:geometry}所示。
\begin{figure}[H]
    \centering
    \includegraphics[width=0.8\textwidth]{images/geometry.png}
    \caption{宏观战场态势图}
    \label{fig:geometry}
\end{figure}
通过观察此图,我们可以发现,无人机群在导弹群前方按照扇形分布,
形成了一个闪现拦截带,无人机FY1距离导弹群最近,
这说明FY1后续可能应该优先对导弹进行拦截。
同时在高度上,无人机群成梯度配置,可以形成低位前伸烟幕拦截网。
\subsection{导弹运动模型构建}
导弹$M_j$以速度 $V_m=300m/s$ 直线飞向假目标$\boldsymbol{P_O}$,其单位方向向量为:
$$
\boldsymbol{n_{M_j}}=\frac{\boldsymbol{P_O}-\boldsymbol{P_{M_j}}\left( 0 \right)}{||\boldsymbol{P_O}-\boldsymbol{P_{M_j}}\left( 0 \right) ||}
$$
\\第$j$枚导弹初始位置为$$\boldsymbol{P}_{M_j}(0) $$
\\导弹在任意时刻 $t$的位置为:
$$\boldsymbol{P}_{M_j}(t)=\boldsymbol{P}_{M_j}(0)+V_m  \boldsymbol{n}_{M_j}$$

\subsection{无人机运动模型构建}
第$i$枚无人机的初始位置为$$\boldsymbol{P}_{FY_i}(0) $$第$i$架无人机飞行速度为 $v_i$,航向角为$\alpha_i$
,其速度向量表示为:
$$
\boldsymbol{v}_{FY_i}=v_i\left( \cos \alpha _i,\sin \alpha _i,0 \right) 
$$
\\无人机在任意时刻 $t$ 的位置为:$$
\boldsymbol{P}_{FY_i}\left( t \right) =\boldsymbol{P}_{FY_i}\left( 0 \right) +\boldsymbol{v}_{FY_i}t
$$
\subsection{烟幕干扰弹的运动模型}
设第$k$枚烟幕弹在 
$t_{drop,k}$ 时刻投放,起爆延时为 
$t_{delay,k}$,重力加速度为$g$
起爆时刻为:
$$t_{burst,k} = t_{drop,k} + t_{delay,k}$$
\\起爆位置为:
$$\boldsymbol{P}_{B,k}=\boldsymbol{P}_{FY_i}(t_{drop,k})+\boldsymbol{v}_{FY_i}t_{delay,k}-\left( 0,0,\frac{1}{2}gt_{delay,k}^2 \right) $$
\subsection{烟幕弹的下沉模型}
烟幕弹起爆后形成半径为 
$R_{cloud}$ 的球形云团,其中心以速度 
$v_{cloud}$ 匀速下沉:

$$\boldsymbol{P}_{S,k}(t)=\boldsymbol{P}_{B,k}-(0,0,v_{cloud}(t-t_{p,k})), \quad t \geq t_{p,k}$$
\subsection{有效遮蔽判定模型}
\subsubsection{几何遮挡条件}
根据题目,可知无人机及导弹的初始位置离真目标(原点为(0,200,0),半径为7,高度为10的圆柱体)所在位置十分遥远,而题中未提到无人机和导弹的视野,那说明其不在题目考虑范围内。我们可以合理假设导弹视野为全范围,即只要在导弹正前方的事物,都处于导弹视野范围内,那对于一个距离十分远,且相对于距离量级来说大小很小的一个真目标,便于进行计算和后续的优化问题,我们可以将其简化为一个质点,进而将遮蔽条件转化为:当烟幕云团中心到“导弹与真目标质心连线”的垂直距离小于等于烟幕半径时,视为有效遮蔽。
\subsubsection{点到线段距离算法}
设导弹位置为点A,真目标位置为点B,烟幕弹中心为点P。向量$\overrightarrow{AB}=B-A$,向量$\overrightarrow{AP}=P-A$。投影系数r计算公式为:$$
r=\frac{\overrightarrow{AP}\cdot\overrightarrow{AB}}{\|\overrightarrow{AB}\|^2}$$
距离分三种情况讨论:
$$D(t)=\begin{cases}
\|\overrightarrow{AP}\| & \text{if } r \leq 0 \text{ (导弹后方)}\\
\|P-B\| & \text{if } r \geq 1 \text{ (目标后方)}\\
\frac{\|\overrightarrow{AP}\times\overrightarrow{AB}\|}{\|\overrightarrow{AB}\|} & \text{if } 0<r<1 \text{ (线段投影内)}
\end{cases}$$
\subsection{遮蔽时间定义}
遮蔽时间定义为:$$
I_j(t)=\begin{cases}1,&\text{if } D_{k,i}(t)\le R_{cloud} \text{ and } t\geq t_{p,k}\\0,&\text{otherwise}\end{cases}
$$其中$D_{k,i}(t)$为第$k$枚烟幕弹对第$j$枚导弹在时刻$t$的点到线段距离。\\
导弹 $j$ 的总有效遮蔽时长为:$$
T_j=\int_0^{T_{end}}{I_j\left( t \right) dt}
$$
在不同问题中,优化目标统一表示为:
$$
\max\text{\ }T_{cover}\ =\ \sum_j{T_j}
$$

% =============================================
% 6. 模型求解
% =============================================
\section{模型求解}

\subsection{问题一求解:固定参数}
问题一针对的是具体场景在给定所有的参数下(烟幕弹的投放时间,起爆时间),求解得出单枚烟幕弹对单枚导弹的有效遮蔽时间,
需要我们利用各个物体的位置函数,结合有效遮挡判据得到有效遮挡时间。
针对本题,飞行策略和投弹策略为题目给定,本题求解核心在于对已知任务的仿真模拟,
所以直接使用时间步长离散化仿真即可解决本题。
\subsubsection{参数代入与场景设定}

根据题目条件,无人机 FY1 的初始位置为
$
\mathbf{P}_{FY_1}(0)=(17800,\,0,\,1800),
$
飞行速度为
$
v_{FY_1}=120\,\mathrm{m/s},
$
航向指向假目标$\boldsymbol{P_O}=(0,0,0)$。
烟幕干扰弹在受领任务后
$
t_{drop}=1.5\,\mathrm{s}
$
时投放,起爆延迟时间为
$
t_{delay}=3.6\,\mathrm{s}.
$
导弹 M1 的初始位置为
$
\mathbf{P}_{M}(0)=(20000,\,0,\,2000),
$
飞行速度为
$
v_M=300\,\mathrm{m/s},
$
并沿直线飞向假目标。
烟幕云团半径取
$R_s=10\,\mathrm{m}$
下沉速度取
$
v_s=3\,\mathrm{m/s}.
$
\subsubsection{运动轨迹计算}
由模型建立中的运动模型,可确定系统中各实体的空间位置。
无人机在投放时刻的空间位置为
$$
\mathbf{P}_{FY_1}
=
\mathbf{P}_{FY_1}(t_{drop}).
$$
烟幕干扰弹的起爆时刻为
$$
t_{burst}=t_{drop}+t_{delay},
$$
其起爆位置为
$$
\mathbf{P}_{burst}
=
\mathbf{P}_{FY_1}(t_{drop})
+
\mathbf{v}_{FY_1}t_{delay}
-
\left(0,\,0,\,\tfrac{1}{2}gt_{delay}^2\right).
$$
起爆后,烟幕云团中心以恒定速度沿竖直方向下沉,其在任意时刻 $t$ 的位置为
$$
\mathbf{P}_{S}(t)
=
\mathbf{P}_{burst}
-
\left(0,\,0,\,v_s(t-t_{burst})\right),
\quad t\ge t_{burst}.
$$
导弹 M1 在任意时刻 $t$ 的位置为
$$
\mathbf{P}_{M}(t)
=
\mathbf{P}_{M}(0)
+
v_M\mathbf{n}_M t,
$$
其中 $\mathbf{n}_M$ 为导弹指向假目标的单位方向向量。

\subsubsection{有效遮蔽时长计算}

为计算烟幕干扰弹对导弹 M1 的有效遮蔽时长,将导弹飞行全过程时间区间 $[0,T_{end}]$ 进行离散化处理。设时间步长为 $\Delta t$,离散时间点为
$$
t_k=k\Delta t,\quad k=0,1,\ldots,N.
$$

在每一离散时刻 $t_k$,计算烟幕云团中心到导弹与真目标连线的最短距离 $d(t_k)$。当满足
$$
d(t_k)\le R_{cloud}
\quad \text{且} \quad
t_k\ge t_{burst}
$$时,认为该时刻烟幕对导弹 M1 实现有效遮蔽。
定义遮蔽指示函数
$$
I(t_k)=
\begin{cases}
1, & d(t_k)\le R_{cloud},\ t_k\ge t_{burst},\\
0, & \text{otherwise}.
\end{cases}
$$

则烟幕干扰弹对导弹 M1 的总有效遮蔽时长可近似表示为
$$
T_{cover}
\approx
\sum_{k=0}^{N} I(t_k)\,\Delta t.
$$


\subsubsection{求解结果:} 经计算,结果如下图\ref{fig:1}所示,烟幕弹起爆时刻:$t=5.10 s$,
烟幕起爆初始坐标:$(17188.00,0.00,1736.50)$有效遮蔽时间段为 $8.1-9.5s$,总时长为 $1.4s$。
    \begin{figure}[H]
        \centering
        \includegraphics[width=0.8\textwidth]{images/resuilt_pro1.png}
        \caption{问题一结果}
        \label{fig:1}
    \end{figure}
\subsection{问题二求解:单机单弹优化}
在问题二中,无人机FY1的任务是通过自身的飞行参数和投弹时机,使得单枚烟雾弹对导弹M1的有效遮蔽时间最大化。
\subsubsection{决策变量}
根据题目要求,无人机一旦受领任务,其速度和航向就确定了且不再调整。因此在进行任务决策时我们需要决策以下四个变量:
飞行速度$v_{FY1}$,
飞行航向角$\alpha$,
投放时刻$t_{drop}$,
延时起爆时间$t_{delay}$,
由此定义出决策向量X:
$$X=[v_{FY1},\alpha,t_{drop},t_{delay}]^T$$


\subsubsection{运动状态方程}



基于问题一的运动学模型,需要将无人机的速度向量参数化。无人机速度向量可表示为:
$$\boldsymbol{v_{FY_1}}=\left[\begin{matrix}v_{FY1}cos{\alpha}\\v_{FY_1}sin{\alpha}\\0\\\end{matrix}\right]$$

投放点位置和起爆点位置更新为用X表示的函数:
$$
\boldsymbol{P}_{drop}\left(X\right)=\boldsymbol{P}_{FY_1}(0)+\boldsymbol{v_{FY_1}} \left(v_{FY_1},\alpha\right)\cdot t_{drop}
$$
$$
\boldsymbol{P}_{burst}\left(X\right)=\boldsymbol{P}_{drop}\left(X\right)+\boldsymbol{v_{FY_1}} \left(v_{FY_1},\alpha\right)\cdot t_{delay}+\left[0,0,-\frac{1}{2}gt_{delay}^2\right]^T
$$

\subsubsection{目标函数}
目标是最大化有效遮蔽时长,利用问题一中定义的遮蔽判定指示函数得出目标函数:
$$max J\left(X\right)=T_{cover}={\int}_0^{T_{end}}I\left(t,X\right)dt$$

\subsubsection{约束条件}
根据题目和对于模型速度特征的分析,约束条件如下:
速度约束:
$$70\le v_{FY_1}\le140$$

时间约束:
$$0<t_{drop}\le12,1\le t_{delay}\le8$$

高度约束:$$
z_{burst}\left(X\right)>0
$$
\subsubsection{求解算法}
由于目标函数涉及复杂的几何运动及分段逻辑,具有高度的非线性和非凸性,且决策变量相对于一般的单变量优化问题具有高自由度。
所以为了解决此类问题,本问题采用双层优化策略,外层使用网格扫描,内层使用粒子群优化算法。
\paragraph{算法整体架构}

外层循环(针对速度,因为速度对优化结果的影响最大):将速度在[70,140]区间内以0.5m/s为步长进行离散化扫描。
内层循环(针对剩余参数):针对每一个固定的速度,使用PSO算法寻找最优组合。
这样分层寻优的好处是将优化问题维度从4维降至3维,降低了搜索难度,降低问题陷入局部最优的概率。
\paragraph{粒子群算法设计}

在内层针对剩余决策变量进行寻优。
设置粒子群规模为N=30; 
粒子速度与位置更新公式:
$$
v_i^{k+1}=wv_i^k+c_1r_1\left(p_{best,i}-x_i^k\right)+c_2r_2\left(g_{best}-x_i^k\right)
$$
$$
x_i^{k+1}=x_i^k+v_i^{k+1}
$$适应度函数:
$$
F\left(x\right)=-T_{cover}\left(x\right)
$$

\paragraph{求解结果:}
经过计算,得到最优投放策略如下:\\
无人机以196.91°的航向角度,72m/s的速度,0s时刻投放,2.4991s后起爆,有效遮蔽时长为4.738s如下图\ref{fig:2}所示。
\begin{figure}[H]
    \centering
    \includegraphics[width=0.8\textwidth]{images/resuilt_pro2_2.png}
    \caption{问题二结果}
    \label{fig:2}
\end{figure}















\subsection{问题三求解:多弹协同策略}
在本题中,无人机 FY1 需要连续投放3枚烟幕弹以对导弹 M1 实施干扰。
与投放一枚烟幕弹不同,三枚烟幕弹的投弹策略涉及时序上的配合,核心问题在对多参数的优化使得烟幕弹的遮蔽时间最大

\subsubsection{决策变量}
根据题目要求,无人机一旦受领任务,其速度和航向就确定了且不再调整。因此在进行任务决策时我们需要决策以下八个变量:
\\飞行速度$$v_{FY_1}$$
飞行航向角$$\alpha$$
投放时刻1$$t_{drop,1} $$
投放时刻2$$t_{drop,2} $$
投放时刻3$$t_{drop,3} $$
延时起爆时间1$$t_{delay,1} $$
延时起爆时间2$$t_{delay,2} $$
延时起爆时间3$$t_{delay,3} $$
由此定义出决策向量X:
$$\mathbf{X}=[v_{FY_1},\alpha,t_{drop,1},t_{drop,2},t_{drop,3},t_{delay,1},t_{delay,2},t_{delay,3}]T  $$

\subsubsection{状态方程更新}
基于问题一的运动学模型,第 $k$枚烟幕弹的运动状态由下式描述
1.	投放点坐标:
 $$P_{drop,k}(\mathbf{X})=B_0+\mathbf{v}_{FY1}(v_{FY1},\alpha)\cdot t_{drop,k}$$
2.	起爆点坐标
 $$P_{burst,k}(\mathbf{X})=P_{drop,k}(\mathbf{X})+\mathbf{v}_{FY1}(v_{FY1},\alpha)\cdot t_{delay,k}+\left[0,0,-\frac{1}{2}gt_{delay,k}^2\right]^T$$
3.	烟幕云团中心轨迹:
起爆时刻$\ t_{burst,k}=t_{drop,k}+t_{delay,k} $后,第$k$个云团中心位置为
 $$P_{S,k}(t)=P_{burst,k}-[0,0,v_{sink}\cdot(t-t_{burst,k})]^T \quad ,\quad t\in[t_{burst,k},t_{burst,k}+20]$$

\subsubsection{目标函数}
由于 3 个烟幕团可能同时存在并产生遮挡效果,总遮蔽时间不能简单求和,而应计算时间轴上的并集。
定义第 $k$枚烟幕弹在$t$时刻的遮蔽指示函数 $I_k(t;\mathbf{X})$:
$$
I_k(t;\mathbf{X})=\begin{cases}
1, & \text{if } D_k(t) \leq R_{cloud} \text{ and } t \in [t_{burst,k}, t_{burst,k}+20] \\
0, & \text{otherwise}
\end{cases}
$$
其中 $D_k(t)$ 为第$k$个烟幕中心到“导弹-目标”视线的距离(计算方法同问题一)。

系统在$t$时刻的遮蔽状态是所有单个烟幕弹的逻辑或
$$
I_{total}(t;\mathbf{X}) = \max\{I_1(t), I_2(t), I_3(t)\}
$$
最终优化目标为最大化总有效遮蔽时长$T_{cover}$:
$$max{J}(\mathbf{X})=T_{cover}=\int_{0}^{T_{end}}I_{total}(t;\mathbf{X})dt$$
\subsubsection{约束条件}
根据题目和对于模型速度特征的分析,约束条件如下:
除了满足问题二中的飞行与高度约束外,本问必须严格满足多枚弹之间的操作间隔约束:\\
投弹间隔约束: 题目要求每架无人机投放两枚烟幕弹至少间隔 1s。
$$t_{drop,k+1}-t_{drop,k}\geq1,\quad k=1,2$$

基本边界约束:
$$
\left\{\begin{matrix}70\le v_{FY_1}\le140\\0<t_{drop,1}<t_{drop,2}<t_{drop,3}\le12\\1\le t_{delay,k}\le8,\quad k=1,2,3\\\end{matrix}\right.
$$
高度约束:
$$z_{burst,k}(\mathbf{X})>0,\quad k=1,2,3$$


\subsubsection{求解算法}
针对决策变量维数增加(由 4 维增至 8 维)且包含时序逻辑约束的问题,
传统的优化算法容易陷入停滞,因此我们采用引入灾变机制的改进粒子群算法求解。
\paragraph{算法整体架构}
不再采用网格扫描,而是直接在 8 维解空间内进行全局寻优。算法通过监测种群的进化状态,动态调整搜索策略,平衡全局探索与局部开发能力.
\paragraph{改进粒子群算法设计}
\begin{itemize}
\item 粒子编码与约束处理:

虽然决策变量定义为绝对时刻,但在算法实现中,为了自动满足间隔约束,对时间参数采用增量编码$(\Delta t_{12},\Delta t_{23})$进行搜索,解码时再转换为绝对时间 $t_{drop}$。
\item 灾变重启机制:\\
针对多波次协同容易陷入局部最优的问题,引入"灾变"策略
\\\textbf{停滞检测:}设置停滞计数器 $\text{stagnation\_counter}$,若全局最优解 $G_{best}$ 连续一定代数未更新,则判定算法陷入停滞。
\\\textbf{灾变操作:}触发灾变,保留种群中前 $10\%$ 的精英粒子(历史最优个体),对其余 $90\%$ 的粒子进行完全重置(位置与速度重新随机化)。该机制能有效打破种群的聚集状态,跳出局部极值陷阱,寻找更优的多弹协同策略。
\item 适应度函数:
对时间轴进行离散化扫描,计算每一时刻系统总遮蔽状态 $I_{sys}(t)$,积分得到总有效时长,取负值作为适应度函数$F(\mathbf{X})=-T_{cover}$进行极小化求解.
\end{itemize}

\subsubsection{求解结果}
根据上述的流程,我们采用灾变机制的改进粒子群算法得到
无人机FY1投放3枚烟幕弹干扰导弹 M1的最优策略见下图\ref{fig:3}
\\对应的最大有效遮挡时间是12.54241s
\\无人机航向方向角为:180.67°
\\无人机速度大小为:95m/s。
\\三枚烟幕弹的具体投放策略见下表\ref{tab:3}。

\begin{table}[htbp]
  \centering
  \caption{Problem\,3 最终策略}
  \label{tab:3}
  \begin{tabular}{ll}
    \toprule
    参数 & 数值 \\
    \midrule
    最大有效遮蔽时长 & 12.54241\,s \\
    无人机速度 & 95\,m/s \\
    无人机航向 & 3.1533\,rad\,(180.67$^\circ$) \\
    \midrule
    \multicolumn{2}{l}{烟弹投放与起爆时刻}\\
    \midrule
    烟弹1投放时刻 & 0.0000\,s \\
    烟弹1起爆时刻 & 0.0000\,s \\
    烟弹2投放时刻 & 2.7509\,s \\
    烟弹2起爆时刻 & 6.4326\,s \\
    烟弹3投放时刻 & 8.0057\,s \\
    烟弹3起爆时刻 & 13.3706\,s \\
    \bottomrule
  \end{tabular}
\end{table}
\begin{figure}[htbp]
  \centering
  \includegraphics[width=0.8\textwidth]{images/resuilt_pro3_2.png}
  \caption{Problem\,3 结果}
  \label{fig:3}
\end{figure}
\subsection{问题四求解:多机协同优化}
在问题四中,任务变更为$FY_1、FY_2、FY_3$ 三架无人机共同拦截导弹 $M1$,每架无人机投放一枚烟幕弹,
核心问题在于协调三架处于不同空间位置的无人机的投弹策略,使得烟幕弹的遮蔽时间最大.




\subsubsection{决策变量}
根据题目要求,无人机一旦受领任务,其速度和航向就确定了且不再调整。
因此在进行任务决策时我们需要决策12维变量:
系统包含 3 个独立的运动实体(无人机)。定义 $i\in{1,2,3}$分别对应 $FY_1, FY_2, FY_3$。每个实体的决策向量$\mathbf{u}_i$ 包含 4 个参数:

$$\mathbf{u}_i=[v_i,\alpha_i,t_{drop,i},t_{delay,i}]$$

全局决策向量X维12维向量:$$
\mathbf{X}=[\mathbf{u}_1,\mathbf{u}_2,\mathbf{u}_3]=[v1,α1,tdrop,1,tdelay,1,…,v3,α3,tdrop,3,tdelay,3]^T$$
\subsubsection{状态方程}
各无人机的初始位置$ \mathbf{Po}\mathbf{s}_{UAV,i} $不同,第$i$架无人机投放的烟幕弹在$t$刻的中心位置 $P_{S,i}(t) $更新公式如下
投放点坐标:
$$P_{drop,i}=\mathbf{Po}\mathbf{s}_{UAV,i}+\mathbf{v}_i(v_i,\alpha_i)\cdot t_{drop,i}$$
起爆点坐标
$$P_{burst,i}=P_{drop,i}+\mathbf{v}_i\cdot t_{delay,i}+\mathbf{G}(t_{delay,i})$$
其中 $\mathbf{G}(t)=[0,0,-\frac{1}{2}gt^2]^T$为重力项。\\
烟幕云团中心轨迹:
$$
P_{S,i}(t)=P_{burst,i}-[0,0,v_{sink}\cdot(t-t_{burst,i})]^T
$$
\subsubsection{目标函数}










由于 3 个烟幕团可能同时存在并产生遮挡效果,总遮蔽时间不能简单求和,而应计算时间轴上的并集。
定义第 i个烟幕云团对目标的遮蔽指示函数$I_i(t)$:$$
I_i(t;\mathbf{X})=\left\{\begin{matrix}1,&\mathrm{if\ \ dist}(P_{S,i}(t),\mathrm{Lin} \mathrm{e}_{M1-Target})\le R_{cloud}\\0,&otherwise\\\end{matrix}\right.
4)$$
系统总遮蔽状态为各子状态的逻辑或:
$$Isys(t)=\max_{i=1,2,3}{I_i(t)}$$
最终优化目标函数定义为最小化代价$J(\mathbf{X})$:
$$\min{J}(\mathbf{X})=-\left(\int_{0}^{T_{end}}I_{sys}(t)dt\right)+\lambda\cdot\sum_{i=1}^{3}{\max(d_{min,i}, R_{cloud})}$$
其中:
第一项为总遮蔽时长(取负求最小),
第二项为惩罚项,$d_{min,i}$ 为第 i枚烟幕弹在其生命周期内距离视线的最近距离,
$\lambda=0.1$为权重系数。当烟幕完全脱靶时,该项通过惩罚距离迫使解向视线靠拢。

\subsubsection{约束条件}
速度约束:
$$70\le v_i\le140,\quad \forall i$$
时间窗约束:我们通过物理知识可知,遮蔽的理想状态是烟幕弹之间无缝接力,所以设置时间窗口约束,减小搜索空间:
$$t_{drop,1}\le15,\quad t_{drop,2}\le35,\quad t_{drop,3}\le55$$
起爆延时约束:
$$1\le t_{delay,i}\le8$$
高度约束:
$$z_{burst,k}(\mathbf{X})>0,\quad k=1,2,3$$
\subsubsection{求解算法}
由于决策变量维数增加为12维,而且各维度之间差异较大,所以第四问我们选择鲁棒性强的差分进化算法(DE)求解

\paragraph{算法流程设计}


\begin{itemize}

    \item 种群初始化:
生成NP=200个个体。针对航向角 $\alpha_i$,基于各无人机到目标的初始视线角 $\theta_{base,i}$ 进行正态分布初始化,范围控制在 $\theta_{base,i}\pm45°$ 内,提高搜索效率

    \item 变异:
    采用 DE/rand/1 策略生成变异向量$\mathbf{V}_g$:
$$\mathbf{V}_{i,g}=\mathbf{X}_{r1,g}+F\cdot(\mathbf{X}_{r2,g}-\mathbf{X}_{r3,g})$$
其中缩放因子F 采用自适应策略,随迭代次数从 0.5 线性衰减,平衡全局探索与局部开发。
    \item 交叉:
   对变异向量与目标向量进行二项式交叉,生成试验向量$ \mathbf{U}_{i,g}$。交叉概率 $CR=0.9$。
    \item 选择:
    计算试验向量与目标向量的适应度,选择适应度更优的个体进入下一代。采用贪婪策略,若试验向量的适应度优于目标向量,则在下一代中保留试验向量:
$$\mathbf{X}_{i,g+1}=\left\{\begin{matrix}\mathbf{U}_{i,g},&\mathrm{if\ }J(\mathbf{U}_{i,g})<J(\mathbf{X}_{i,g})\\\mathbf{X}_{i,g},&\mathrm{otherwise}\\\end{matrix}\right.$$
    \item 终止条件:
    	最大迭代次数$G_{max}=800$ 或适应度不再显著提升
\end{itemize}

\subsubsection{求解结果}

我们通过差分进化算法(DE)用matlab求解得到了最大的总遮蔽有效时间15s,
并且烟幕弹的遮蔽时间衔接良好,提供较好的遮蔽效果。对应的三机协同投放策略如下表\ref{tab:4}所示,效果如下图\ref{fig:4}所示。

\begin{table}[H]
\centering
\label{tab:4}
\renewcommand{\arraystretch}{1.25}
\setlength{\tabcolsep}{6pt}
\caption{问题\,4 三机协同立体封锁最终策略}
\begin{tabular}{@{}lccccccc@{}}
\toprule
机号 & 初始位置\,(m) & 速度\,(m/s) & 航向\,($^\circ$) & 投放时刻\,(s) & 延时\,(s) & 起爆时刻\,(s) \\
\midrule
FY1 & (17800,\,0,\,1800) & 78.65 & 178.42 & 0.67 & 2.87 & 3.55 \\
FY2 & (12000,\,1400,\,1400) & 119.78 & -129.29 & 6.69 & 7.84 & 14.53 \\
FY3 & (6000,\,-3000,\,700) & 135.11 & 106.94 & 18.00 & 6.09 & 24.09 \\
\bottomrule
\end{tabular}
\end{table}

\begin{figure}[H]
    \centering
    \includegraphics[width=0.8\textwidth]{images/pro4_2.png}
    \caption{问题\,4 三机协同立体封锁结果}
    \label{fig:4}
\end{figure}

\subsection{问题五求解:多机多目标协同拦截}
在问题五中,作战场景扩展为5架无人机$FY_1-FY_5$协同拦截3枚来袭导弹$M_1--M_3$。
每架无人机最多可携带3枚烟幕弹。问题的核心在于如何在复杂的空间分布下,合理分配无人机的飞行策略和弹药资源,使得对所有导弹的总有效遮蔽时间最大化。
\subsubsection{决策变量体系}
考虑到问题规模显著扩大,且受到前几问优化算法的启发,我们将决策变量分为战略层(飞行参数)和战术层(投弹参数)。
战略决策变量:定义全局飞行决策向量,包含5架无人机的速度和航向:
$$X_{fly}=[v_1,\alpha_1,v_2,\alpha_2,\ldots,v_5,\alpha_5]^T$$
战术决策变量:对于第i架无人机,其携带的第k枚烟幕弹$(k=1,2,3)$的投放参数由算法在内层生成,不直接作为外层优化变量,但需满足物理约束:
$$u_{i,k}=\left[t_{drop}^{\left(i,k\right)},t_{delay}^{\left(i,k\right)}\right]$$
\subsubsection{状态空间}
多导弹运动模型的建立参考第一问,因为每个导弹的飞行特性为一致,只有初始位置不同。
多无人机运动模型亦可由之前类比得到。
\subsubsection{目标函数}
由于存在3个独立的来袭目标,系统总效益定义为所有导弹被有效遮蔽时长的总和:
$$S_j\left(t\right)=\bigvee_{i=1}^{5}\bigvee_{k=1}^{3}I_{i,k,j}\left(t\right)$$

最终优化目标为最大化总效益J:
$$\max J(X_{fly})=\sum_{j=1}^{3}\left(\int_0^{T_{end}}S_j(t)dt\right)$$
\subsubsection{约束条件}
最大载弹量约束:每架无人机投放数量N<=3;
\\投弹间隔约束:
$$|t_{drop}^{\left(i,k\right)}-_{drop}^{\left(i,k-1\right)}|\geq1,\forall k>1
    $$
飞行边界约束:
$$70\le v_i\le140,0\le\alpha_i\le2\pi$$

\subsubsection{求解算法}
针对5机3弹的高维组合优化问题,依旧采用双层优化结构。外层使用遗传算法搜索无人机的最优飞行参数,内层使用贪心启发式算法快速计算给定航迹下的最优投弹方案。

\begin{itemize}
    \item 外层:遗传算法全局寻优
    将5架无人机的速度和航向编码为一条染色体。采用实数编码,操作算子采用迷你二进制交叉和多项式变异。
实现编码$X_{fly}$,则适应度函数为:
$$Fitness\left(X_{fly}\right)=-J\left(X_{fly}\right)$$

    \item 内层:贪心策略解算投弹方案
    对于种群中每一个体确定的飞行轨迹,内层算法通过以下步骤计算适应度:
\\Step 1: 生成候选打击集
遍历第i架无人机飞行路径上的离散时间点t,对于每一个t,遍历可能的延时起爆时间,检测生成的烟幕云团能否拦截 M1, M2, M3 中的任意一枚。若能拦截,记录该候选方案。
\\Step 2: 贪心选择 
对候选方案集按遮蔽时长降序排列。依次选择遮蔽贡献最大的方案加入最终策略,需满足:该无人机已选弹药数$<3$;与该无人机已选方案的时间间隔$>=1s$。
\\Step 3: 计算并集效益
汇总所有无人机的投弹方案,分别计算对M1, M2, M3的遮蔽时间区间并集,求和得到总分。
此方法的优势在于将复杂的时序配合问题解耦:外层负责“站位”(规划航线),内层负责“输出”(寻找最佳开火时机),在保证解的质量的同时大幅降低了计算复杂度。

\end{itemize}

\subsubsection{求解结果}

通过该算法求解,得到5机3弹协同拦截3枚导弹的最优策略如下表\ref{tab:5}所示,总有效遮蔽时间为28.763s,效果如图\ref{fig:5}所示。
\begin{table}[htbp]
\centering
\label{tab:5}
\renewcommand{\arraystretch}{1.2}
\setlength{\tabcolsep}{4.5pt}
\caption{五机协同的详细方案}
\rowcolors{3}{gray!10}{white}
\begin{tabular}{@{}lcccccc@{}}
\toprule
\rowcolor{gray!25}
\textbf{Drone} & \textbf{BombID} & \textbf{DropTime\,(s)} & \textbf{ExpTime\,(s)} & \textbf{Target} & \textbf{CoverDuration\,(s)} & \textbf{Interval} \\
\midrule
\multicolumn{7}{@{}l}{\cellcolor{blue!15}\bfseries FY1: 88.19\,m/s, 178.35$^\circ$(3.11\,rad)}\\
FY1 & \#1 & 0.00 & 3.00 & M1 & 3.80 & [3.0--6.8]\\
FY1 & \#2 & 1.00 & 4.00 & M  & 3.80 & [4.6--8.4]\\
\midrule
\multicolumn{7}{@{}l}{\cellcolor{yellow!15}\bfseries FY2: 110.93\,m/s, 283.65$^\circ$(4.95\,rad)}\\
FY2 & \#1 & 6.00 & 9.00  & M2 & 4.00 & [9.0--13.0]\\
FY2 & \#2 & 8.00 & 13.00 & M1 & 0.80 & [24.8--25.6]\\
\midrule
\multicolumn{7}{@{}l}{\cellcolor{orange!15}\bfseries FY3: 108.05\,m/s, 121.72$^\circ$(2.12\,rad)}\\
FY3 & \#1 & 28.00 & 35.00 & M2 & 3.40 & [37.8--41.2]\\
FY3 & \#2 & 25.00 & 32.00 & M3 & 3.00 & [35.6--38.6]\\
FY3 & \#3 & 26.00 & 33.00 & M  & 1.60 & [49.4--51.0]\\
\midrule
\multicolumn{7}{@{}l}{\cellcolor{red!15}\bfseries FY4: 135.80\,m/s, 262.62$^\circ$(4.58\,rad)}\\
FY4 & \#1 & 1.00 & 12.00 & M2 & 4.40 & [15.0--19.4]\\
\midrule
\multicolumn{7}{@{}l}{\cellcolor{green!15}\bfseries FY5: 134.90\,m/s, 123.38$^\circ$(2.15\,rad)}\\
FY5 & \#1 & 13.00 & 18.00 & M1 & 4.00 & [19.6--23.6]\\
\bottomrule
\end{tabular}
\end{table}
\begin{figure}[H]
    \centering
    \includegraphics[width=0.8\textwidth]{images/pro5_2.png}
    \caption{五机协同的最优方案}
    \label{fig:5}
\end{figure}




% =============================================
% 7. 结果分析与评价
% =============================================
\section{模型评价}
\subsection{模型的优点}
\begin{itemize}
    \item 考虑了烟幕弹的物理沉降特性,符合实际战场环境。
    \item 算法具有较强的通用性,可扩展至更多数量的无人机集群。
\end{itemize}

\subsection{模型的改进方向}
\begin{itemize}
    \item 当前模型假设导弹不机动,未来可增加导弹末端机动的对抗策略。
\end{itemize}

% =============================================
% 参考文献
% =============================================
\begin{thebibliography}{99}
    \bibitem{ref1} 作者. 题目. 期刊名, 年份, 卷(期): 页码.
    \bibitem{ref2} 张三. 无人机协同作战理论. 北京: 科学出版社, 2020.
    \bibitem{ref3} CUMCM 组委会. 202x年全国大学生数学建模竞赛赛题.
\end{thebibliography}

% =============================================
% 附录
% =============================================
\newpage
\section*{附录:主要代码}
\subsection*{MATLAB 轨迹仿真代码}
\begin{lstlisting}[language=Matlab,caption={Problem 1}]
clc; clear; close all;
%% 1. 参数设置 
% 烟幕弹参数
R_smoke = 10;           % 有效遮蔽半径 (阈值)
Time_smoke_last = 25;   % 延长一点仿真时间以便看全曲线
V_smoke_sink = 3;       % 烟雾团下沉速度 
% 物理常数
g = 9.8; 
% 目标位置
Pos_FakeTarget = [0, 0, 0];          % 假目标 (导弹飞行的目的地)
Pos_TrueTarget_Center = [0, 200, 5]; % 真目标 (我们要保护的对象)
%% 2. 计算烟幕弹起爆初始位置 
% FY1 初始状态
Pos_FY1_0 = [17800, 0, 1800];
V_FY1 = 120; 
% 时间节点
t_drop = 1.5;          % 投放时刻
t_delay = 3.6;         % 延时时长
t_pop = t_drop + t_delay; % 起爆时刻 (5.1s)
% 1. 投放点位置 (FY1 飞行 1.5s)
Pos_Drop = Pos_FY1_0 + [-1, 0, 0] * V_FY1 * t_drop;
% 2. 起爆点初始位置 (平抛运动 3.6s)
Delta_X = -1 * V_FY1 * t_delay; 
Delta_Z = -0.5 * g * t_delay^2;
Pos_Smoke_Init = Pos_Drop + [Delta_X, 0, Delta_Z];
fprintf('烟幕弹起爆时刻: t = %.2f s\n', t_pop);
fprintf('烟幕起爆初始坐标: (%.2f, %.2f, %.2f)\n', Pos_Smoke_Init);
%% 3. 计算有效遮蔽时长 (并记录数据用于画图)
% M1 初始状态
Pos_M1_0 = [20000, 0, 2000];
V_M1 = 300;
Vec_M1 = Pos_FakeTarget - Pos_M1_0; % 方向指向假目标
Dist_M1_Total = norm(Vec_M1);
Dir_M1 = Vec_M1 / Dist_M1_Total;
% 时间积分设置
dt = 0.05; % 稍微调大一点点dt,画图不需要太密集
valid_time = 0; 
% --- 初始化数据记录数组 ---
t_log = [];    % 记录时间
dist_log = []; % 记录距离
fprintf('正在模拟动态遮挡过程...\n');
% 从 t=0 开始记录导弹位置,但 t<t_pop 时距离设为 NaN 或 到初始点的距离
Total_Sim_Time = 30; % 总仿真时长 (对应画图横坐标 0-30s)
for t_current = 0 : dt : Total_Sim_Time
    
    % A. 更新导弹位置 (一直飞)
    dist_flown = V_M1 * t_current;
    if dist_flown >= Dist_M1_Total
        Current_M1_Pos = Pos_FakeTarget; % 已经击中假目标
    else
        Current_M1_Pos = Pos_M1_0 + Dir_M1 * dist_flown;
    end
    
    % B. 更新烟雾位置 & 计算距离
    % 只有在时间超过起爆时间后,烟雾才存在
    if t_current >= t_pop
        t_relative = t_current - t_pop;
        
        % 烟雾下沉
        if t_relative <= Time_smoke_last
            Sink_Dist = V_smoke_sink * t_relative;
            Current_Smoke_Pos = Pos_Smoke_Init - [0, 0, Sink_Dist];
            
            % --- 计算遮挡距离 (核心算法) ---
            P1 = Current_M1_Pos;
            P2 = Pos_TrueTarget_Center; 
            Q = Current_Smoke_Pos; 
            
            v = P2 - P1;
            w = Q - P1;
            c1 = dot(w, v);
            c2 = dot(v, v);
            
            if c1 <= 0
                dist = norm(Q - P1);
            elseif c2 <= c1
                dist = norm(Q - P2);
            else
                b = c1 / c2;
                Pb = P1 + b * v;
                dist = norm(Q - Pb);
            end
            
            % 统计有效时间
            if dist <= R_smoke
                valid_time = valid_time + dt;
            end
            
        else
            % 烟雾消失了
            dist = NaN; 
        end
    else
        % 还没起爆,计算导弹到"未来起爆点"的距离 
        dist = norm(Pos_Smoke_Init - Current_M1_Pos); 
    end
    
    % C. 记录数据
    t_log = [t_log, t_current];
    dist_log = [dist_log, dist];
end
fprintf('--------------------------------------\n');
fprintf('>>> 最终结果:有效遮蔽时长为: %.4f 秒 <<<\n', valid_time);
%% 4. 绘图 (完全复刻之前的样式)
figure('Color', 'w', 'Name', '烟幕干扰弹有效遮蔽时间分析');
hold on; grid on; box on;
% --- A. 准备填充区域数据 ---
% 找到距离小于阈值且时间大于起爆时间的索引
mask = (dist_log <= R_smoke) & (t_log >= t_pop);
if any(mask)
    % 提取需要填充的时间段和距离
    t_fill_data = t_log(mask);
    dist_fill_data = dist_log(mask);
    
    % 构造多边形:上边缘是阈值线,下边缘是实际曲线
    % 顶点顺序:[左上 -> 曲线点集 -> 右上]
    X_poly = [t_fill_data, fliplr(t_fill_data)];
    Y_poly = [dist_fill_data, ones(size(dist_fill_data)) * R_smoke];
    
    % 绘制淡黄色填充
    h_area = fill(X_poly, Y_poly, [1 1 0.8], 'EdgeColor', 'none', 'FaceAlpha', 0.5);
    
    % 找到有效结束时间 (遮蔽区间的最后一个时刻)
    t_end_effective = t_fill_data(end);
else
    t_end_effective = t_pop; % 如果没遮住,结束时间就标在起爆点
    h_area = plot(NaN, NaN, 'y'); % 占位符防止legend报错
end
% --- B. 绘制主要曲线 ---
h_dist = plot(t_log, dist_log, 'b-', 'LineWidth', 2); 
% --- C. 绘制辅助线 (阈值、起爆、结束) ---
% 1. 阈值红虚线
h_thresh = plot([0, 30], [R_smoke, R_smoke], 'r--', 'LineWidth', 1); 
% 2. 起爆时间绿虚线
h_start = xline(t_pop, 'g--', 'LineWidth', 1.5); 
% 3. 有效结束绿虚线 (如果有遮蔽的话)
if any(mask)
    h_end = xline(t_end_effective, 'g--', 'LineWidth', 1.5);
    % 标注文字:有效结束
    text(t_end_effective + 0.5, 60, '有效结束', 'Color', 'g', 'Rotation', 90, 'FontSize', 10);
else
    h_end = plot(NaN,NaN,'g--'); % 占位
end
% --- D. 添加文字标注 ---
% 起爆时间
text(t_pop - 0.8, 60, '起爆时间', 'Color', 'g', 'Rotation', 90, 'FontSize', 10);
text(25, R_smoke + 2, ['阈值 ' num2str(R_smoke) 'm'], 'Color', 'r', 'FontSize', 10);
% --- E. 图例与坐标轴设置 ---
title('烟幕干扰弹有效遮蔽时间分析', 'FontSize', 12);
xlabel('时间 (s)', 'FontSize', 11);
ylabel('到导弹路径的垂直距离 (m)', 'FontSize', 11);
xlim([0, 30]);
ylim([0, 70]); % 根据你的图片要求定在70,如果数据很大可以改大
legend([h_dist, h_thresh, h_start, h_end, h_area], ...
    {'距离曲线', '阈值', '起爆时间', '有效结束', '有效遮蔽区间'}, ...
    'Location', 'north', 'Orientation', 'vertical');
hold off;
\end{lstlisting}
\begin{lstlisting}[language=Matlab , caption=Problem 2]
    function Full_Range_PSO_Sweep()
    clc; close all;

    %% 1. 场景基础参数
    Env.g = 9.8; 
    Env.Pos_FakeTarget = [0, 0, 0];           
    Env.Pos_TrueTarget = [0, 200, 5];         
    Env.Pos_M1_Init = [20000, 0, 2000];       
    Env.V_M1 = 300;                           
    Env.Vec_M = Env.Pos_FakeTarget - Env.Pos_M1_Init;
    Env.Dir_M1 = Env.Vec_M / norm(Env.Vec_M);         
    Env.Dist_Total_M1 = norm(Env.Vec_M);
    Env.Pos_FY1_Init = [17800, 0, 1800];      
    Env.V_smoke_sink = 3;      
    Env.R_smoke = 10;           
    Env.Time_smoke_last = 25;   
    
    % === 仿真精度设置 ===
    % 在PSO搜索阶段使用 1ms 精度,足够区分优劣且速度可控
    Env.dt = 0.001; 

    %% 2. 扫描设置
    v_scan_list = 70 : 0.5 : 140; % 您的要求:70-140,步长0.5
    num_scan = length(v_scan_list);
    
    % 结果存储
    results_score = zeros(num_scan, 1);
    results_params = zeros(num_scan, 3); % [航向, 投放, 延时]

    fprintf('======================================================\n');
    fprintf('      FY1 全速域 PSO 暴力扫描 (Grid-PSO)\n');
    fprintf('      速度: 70-140 m/s | 步长: 0.5 m/s | 核心: PSO\n');
    fprintf('======================================================\n');
    
    try
        if isempty(gcp('nocreate')), parpool; end
        fprintf('>>> 并行计算池已就绪,全速运算中...\n');
    catch
        fprintf('>>> 并行启动失败,切换为单核运算...\n');
    end

    timer_start = tic;

    %% 3. 并行扫描循环
    parfor i = 1 : num_scan
        v_curr = v_scan_list(i);
        
        % --- PSO 配置 ---
        % 在固定的速度下,寻找最佳的 [航向, 投放, 延时]
        % 变量维度: 3
        % 搜索范围: 
        %   航向: 基准 +/- 30度
        %   投放: 0 - 12秒
        %   延时: 1 - 8秒
        
        % 计算基准航向
        Vec_Base = Env.Pos_TrueTarget - Env.Pos_FY1_Init;
        Base_Angle = rad2deg(atan2(Vec_Base(2), Vec_Base(1)));
        
        LB = [Base_Angle-30, 0, 1.0];
        UB = [Base_Angle+30, 12, 8.0];
        
        % 运行内置的微型 PSO
        % 粒子数: 30 (对3维问题足够)
        % 迭代: 50
        [best_x, best_val] = Run_Micro_PSO(v_curr, LB, UB, Env);
        
        % 记录
        results_score(i) = best_val;
        results_params(i, :) = best_x;
        
        if mod(i, 10) == 0
            fprintf('    ...完成速度 %.1f m/s (最佳遮蔽: %.4fs)\n', v_curr, best_val);
        end
    end
    
    total_time = toc(timer_start);
    fprintf('\n>>> 扫描完成!总耗时: %.2f 秒\n', total_time);

    %% 4. 结果分析与可视化
    [max_score, idx_best] = max(results_score);
    best_v = v_scan_list(idx_best);
    best_p = results_params(idx_best, :); 
    
    fprintf('\n======================================================\n');
    fprintf('                全局最优解 (Global Optimum)\n');
    fprintf('======================================================\n');
    fprintf('>>> 冠军速度: %.1f m/s\n', best_v);
    fprintf('>>> 极限遮蔽: %.5f 秒\n', max_score);
    fprintf('>>> 战术参数: 航向 %.2f度 | 投放 %.4fs | 延时 %.4fs\n', best_p(1), best_p(2), best_p(3));

    % --- 绘制“速度-性能”全景图 ---
    figure('Color', 'w', 'Name', 'Speed vs Shield Time');
    plot(v_scan_list, results_score, 'b-', 'LineWidth', 1.5); hold on;
    scatter(best_v, max_score, 100, 'r', 'filled', 'p');
    
    % 标记关键点
    xlabel('飞行速度 (m/s)', 'FontSize', 12);
    ylabel('最大有效遮蔽时长 (s)', 'FontSize', 12);
    title('全速域战术性能分布图 (PSO搜索)', 'FontSize', 14);
    grid on;
    
    % 自动寻找局部峰值 (比如高速区的峰值)
    [pks, locs] = findpeaks(results_score);
    if ~isempty(pks)
        % 标记除了全局最优外的另一个显著峰值(如果有)
        sorted_pks = sort(pks, 'descend');
        if length(sorted_pks) > 1 && sorted_pks(2) > 4.0
            second_pk = sorted_pks(2);
            % 找到对应速度
            idx_sec = find(results_score == second_pk, 1);
            sec_v = v_scan_list(idx_sec);
            if abs(sec_v - best_v) > 10 % 确保两个峰隔得够远
                plot(sec_v, second_pk, 'ko', 'MarkerFaceColor', 'y');
                text(sec_v, second_pk+0.1, sprintf('次优解\n%.1fm/s', sec_v), 'HorizontalAlignment', 'center');
            end
        end
    end
    
    text(best_v, max_score + 0.15, sprintf('全局最优\nV=%.1f\nT=%.4fs', best_v, max_score), ...
        'Color', 'r', 'HorizontalAlignment', 'center', 'FontWeight', 'bold');

    %% 5. 最终验证 (使用 0.0001s 航天级精度复核冠军参数)
    fprintf('\n>>> 正在使用 0.1ms 精度复核冠军方案...\n');
    final_check_score = Tactical_Sim_Engine(best_v, best_p(1), best_p(2), best_p(3), 0.0001, Env);
    fprintf('>>> 最终复核得分: %.6f 秒\n', -final_check_score); % 注意引擎返回负值
    
    % 绘制冠军轨迹
    Plot_3D_Final(best_v, best_p, Env);
end

%% --- 内部微型 PSO 求解器 ---
function [best_pos, best_val] = Run_Micro_PSO(v, lb, ub, Env)
    % 粒子群参数
    n_part = 30;
    n_iter = 50;
    w = 0.6; c1 = 1.5; c2 = 1.5;
    n_vars = 3;
    
    % 初始化
    pos = repmat(lb, n_part, 1) + rand(n_part, n_vars) .* repmat(ub-lb, n_part, 1);
    
    % [种子注入] 针对不同速度段注入经验值,加速收敛
    if v < 90
        pos(1,:) = [176.88, 0.01, 2.5]; % 低速经验
    else
        pos(1,:) = [178.46, 0.01, 3.3]; % 高速经验
    end
    
    vel = zeros(n_part, n_vars);
    
    pbest_pos = pos;
    pbest_val = zeros(n_part, 1);
    
    gbest_pos = zeros(1, n_vars);
    gbest_val = -1e9; % 初始化为很小的值
    
    % 评估初始种群
    for i = 1:n_part
        % 注意:Sim引擎返回的是负值(为了fminsearch),这里我们取相反数变回正值方便比较
        val = -Tactical_Sim_Engine(v, pos(i,1), pos(i,2), pos(i,3), Env.dt, Env);
        pbest_val(i) = val;
        if val > gbest_val
            gbest_val = val;
            gbest_pos = pos(i,:);
        end
    end
    
    % 迭代
    for t = 1:n_iter
        for i = 1:n_part
            r1 = rand(1, n_vars); r2 = rand(1, n_vars);
            vel(i,:) = w*vel(i,:) + c1*r1.*(pbest_pos(i,:)-pos(i,:)) + c2*r2.*(gbest_pos-pos(i,:));
            pos(i,:) = pos(i,:) + vel(i,:);
            
            % 边界限制
            pos(i,:) = max(pos(i,:), lb);
            pos(i,:) = min(pos(i,:), ub);
            
            val = -Tactical_Sim_Engine(v, pos(i,1), pos(i,2), pos(i,3), Env.dt, Env);
            
            if val > pbest_val(i)
                pbest_val(i) = val;
                pbest_pos(i,:) = pos(i,:);
            end
            if val > gbest_val
                gbest_val = val;
                gbest_pos = pos(i,:);
            end
        end
    end
    
    best_pos = gbest_pos;
    best_val = gbest_val;
end

%% --- 仿真核函数 (Cost Function) ---
function score = Tactical_Sim_Engine(v, a, td, ty, dt, Env)
    % 返回负值用于最小化,正值代表遮蔽时长
    
    % 几何解算
    ang_rad = deg2rad(a);
    Dir_Vec = [cos(ang_rad), sin(ang_rad), 0];
    Vel_Vec = Dir_Vec * v;
    
    P_Drop = Env.Pos_FY1_Init + Vel_Vec * td;
    P_Pop = P_Drop + Vel_Vec * ty;
    P_Pop(3) = P_Pop(3) - 0.5 * Env.g * ty^2;
    
    % 剪枝:地下
    if P_Pop(3) < 0 
        score = 0; return;
    end
    
    t_pop = td + ty;
    t_start = max(0, t_pop);
    t_end = min(Env.Dist_Total_M1/Env.V_M1, t_pop + Env.Time_smoke_last);
    
    if t_start >= t_end
        score = 0; return;
    end
    
    % 矢量化计算
    t_vec = t_start : dt : t_end;
    if isempty(t_vec), score=0; return; end
    
    d_m_vec = Env.V_M1 * t_vec;
    P_M_mat = Env.Pos_M1_Init' + Env.Dir_M1' * d_m_vec; 
    P_Smk_mat = P_Pop' - [0;0;Env.V_smoke_sink] * (t_vec - t_pop);
    
    V_LOS_mat = Env.Pos_TrueTarget' - P_M_mat;
    W_mat = P_Smk_mat - P_M_mat;
    
    c1 = sum(W_mat .* V_LOS_mat, 1);
    c2 = sum(V_LOS_mat .* V_LOS_mat, 1);
    
    b = c1 ./ c2;
    
    % 垂足修正
    Pb = P_M_mat + V_LOS_mat .* b;
    idx_less = b < 0; if any(idx_less), Pb(:, idx_less) = P_M_mat(:, idx_less); end
    idx_more = b > 1; if any(idx_more), Pb(:, idx_more) = repmat(Env.Pos_TrueTarget', 1, sum(idx_more)); end
    
    dists_sq = sum((P_Smk_mat - Pb).^2, 1);
    
    count = sum(dists_sq <= Env.R_smoke^2);
    score = -(count * dt); % 返回负值
end

%% --- 绘图 ---
function Plot_3D_Final(v, params, Env)
    a=params(1); td=params(2); ty=params(3);
    ang_rad = deg2rad(a);
    Vel_Vec = [cos(ang_rad), sin(ang_rad), 0] * v;
    P_Drop = Env.Pos_FY1_Init + Vel_Vec * td;
    P_Pop = P_Drop + Vel_Vec * ty;
    P_Pop(3) = P_Pop(3) - 0.5 * Env.g * ty^2;
    
    figure('Color','w'); hold on; grid on; axis equal; view([-20, 25]);
    plot3(Env.Pos_TrueTarget(1), Env.Pos_TrueTarget(2), Env.Pos_TrueTarget(3), 'go', 'MarkerSize', 10, 'MarkerFaceColor','g');
    plot3(Env.Pos_FakeTarget(1), Env.Pos_FakeTarget(2), Env.Pos_FakeTarget(3), 'kp', 'MarkerSize', 10, 'MarkerFaceColor','k');
    plot3([Env.Pos_FY1_Init(1) P_Drop(1)],[Env.Pos_FY1_Init(2) P_Drop(2)],[Env.Pos_FY1_Init(3) P_Drop(3)],'b-');
    plot3([P_Drop(1) P_Pop(1)],[P_Drop(2) P_Pop(2)],[P_Drop(3) P_Pop(3)],'k:');
    plot3(P_Pop(1), P_Pop(2), P_Pop(3), 'rh', 'MarkerFaceColor','r','MarkerSize',14);
    line([Env.Pos_M1_Init(1) Env.Pos_TrueTarget(1)], [Env.Pos_M1_Init(2) Env.Pos_TrueTarget(2)], [Env.Pos_M1_Init(3) Env.Pos_TrueTarget(3)], 'Color', [0.7 0.7 0.7], 'LineStyle', '--');
    title(['Global Optimal: V=' num2str(v) ' m/s']);
    legend('True','Fake','FY1','Bomb','Pop','LOS');
end
\end{lstlisting}
\begin{lstlisting}[language=Matlab, caption=Problem 3]
clc; clear; close all;

%% 1. 参数初始化
% 粒子群算法参数
n_particles = 200;      % 增加粒子数以覆盖更大的搜索空间(8维)
n_iterations = 300;     % 增加迭代次数
c1 = 2.0; c2 = 2.0;
w_max = 0.9; w_min = 0.4; 

% 变量设计 (8维): 
% x(1): 速度 v (70-140)
% x(2): 航向 theta (0-2pi)
% x(3): 第1枚投放时刻 t1 (0-60)
% x(4): 第1枚与第2枚的时间间隔 dt12 (1-30) -> 保证了间隔>=1s
% x(5): 第2枚与第3枚的时间间隔 dt23 (1-30) -> 保证了间隔>=1s
% x(6): 第1枚起爆延时 delay1 (0-15)
% x(7): 第2枚起爆延时 delay2 (0-15)
% x(8): 第3枚起爆延时 delay3 (0-15)

lb = [70,  0,    0,  1,  1,  0,  0,  0];
ub = [140, 2*pi, 60, 20, 20, 15, 15, 15];

% 速度限制
v_max = 0.15 * (ub - lb);
v_min = -v_max;

%% 2. 种群初始化
particles = repmat(lb, n_particles, 1) + rand(n_particles, 8) .* (ub - lb);
velocities = zeros(n_particles, 8);

pbest_pos = particles;
pbest_val = inf(n_particles, 1);
gbest_pos = zeros(1, 8);
gbest_val = inf;

stagnation_counter = 0; % 停滞计数器,用于判断是否陷入局部最优

%% 3. PSO 主循环
disp('开始多烟幕弹协同优化 (引入灾变机制防止局部最优)...');
tic; 

for iter = 1:n_iterations
    w = w_max - (w_max - w_min) * iter / n_iterations;
    
    current_gbest_improved = false;
    
    for i = 1:n_particles
        current_x = particles(i, :);
        current_x(1) = round(current_x(1)); % 速度取整
        
        % 边界处理
        current_x = max(current_x, lb);
        current_x = min(current_x, ub);
        
        % 计算适应度 (核心修改:支持3枚弹)
        fitness = calculate_fitness_multi(current_x);
        
        % 更新个体最优
        if fitness < pbest_val(i)
            pbest_val(i) = fitness;
            pbest_pos(i, :) = current_x;
        end
        
        % 更新全局最优
        if fitness < gbest_val
            gbest_val = fitness;
            gbest_pos = current_x;
            current_gbest_improved = true;
        end
    end
    
    % --- 灾变机制 (防止陷入局部最优) ---
    if current_gbest_improved
        stagnation_counter = 0;
    else
        stagnation_counter = stagnation_counter + 1;
    end
    
    % 如果连续20代没有进步,触发“小灾变”:重置部分粒子
    if stagnation_counter > 20 && iter < n_iterations - 50
        disp(['    >> 检测到局部最优停滞,触发灾变重置 (Iter ' num2str(iter) ')']);
        % 保留前 10% 的精英,重置其余 90%
        n_keep = round(n_particles * 0.1);
        [~, sorted_idx] = sort(pbest_val);
        for k = n_keep+1 : n_particles
            idx = sorted_idx(k);
            particles(idx, :) = lb + rand(1, 8) .* (ub - lb);
            velocities(idx, :) = zeros(1, 8);
            pbest_val(idx) = inf; % 重置历史记忆
        end
        stagnation_counter = 0;
    end
    
    % 粒子更新公式
    for i = 1:n_particles
        r1 = rand(1, 8);
        r2 = rand(1, 8);
        
        velocities(i,:) = w * velocities(i,:) ...
            + c1 * r1 .* (pbest_pos(i,:) - particles(i,:)) ...
            + c2 * r2 .* (gbest_pos - particles(i,:));
        
        velocities(i,:) = max(min(velocities(i,:), v_max), v_min);
        particles(i,:) = particles(i,:) + velocities(i,:);
        
        % 变异操作 (5%概率)
        if rand < 0.05 
            dim = randi(8);
            particles(i, dim) = lb(dim) + rand * (ub(dim) - lb(dim));
        end
        particles(i,:) = max(min(particles(i,:), ub), lb);
    end
    
    % 进度显示
    if mod(iter, 10) == 0 || iter == 1
        real_time = 0;
        if gbest_val < -1000, real_time = -(gbest_val + 10000); end
        fprintf('迭代 %d/%d, 当前最佳遮蔽时长: %.5f 秒 (停滞: %d)\n', ...
            iter, n_iterations, real_time, stagnation_counter);
    end
end
toc;

%% 4. 结果解析与输出
best_params = gbest_pos;
best_params(1) = round(best_params(1));

% 解码时间参数
t_drop1 = best_params(3);
t_drop2 = t_drop1 + best_params(4);
t_drop3 = t_drop2 + best_params(5);
t_delay1 = best_params(6);
t_delay2 = best_params(7);
t_delay3 = best_params(8);

% 最终高精度验证
final_time = calculate_masking_pure_multi(best_params, 0.00001);

fprintf('\n---------------- 最终优化结果 (Problem 3) ----------------\n');
fprintf('最大有效遮蔽时长: %.5f 秒\n', final_time);
fprintf('\n策略详情:\n');
fprintf('无人机速度: %d m/s\n', best_params(1));
fprintf('无人机航向: %.4f rad (%.2f 度)\n', best_params(2), rad2deg(best_params(2)));
fprintf('\n[烟弹1] 投放时刻: %6.4f s, 起爆延时: %6.4f s, 起爆时刻: %6.4f s\n', t_drop1, t_delay1, t_drop1+t_delay1);
fprintf('[烟弹2] 投放时刻: %6.4f s, 起爆延时: %6.4f s, 起爆时刻: %6.4f s\n', t_drop2, t_delay2, t_drop2+t_delay2);
fprintf('[烟弹3] 投放时刻: %6.4f s, 起爆延时: %6.4f s, 起爆时刻: %6.4f s\n', t_drop3, t_delay3, t_drop3+t_delay3);
fprintf('注:投放间隔分别为 %.4f s 和 %.4f s (均满足>=1s约束)\n', best_params(4), best_params(5));


%% ================= 子函数定义区域 =================

function fitness = calculate_fitness_multi(x)
    % 1. 解码所有参数
    v = x(1); theta = x(2);
    t_d1 = x(3); 
    t_d2 = t_d1 + x(4); % 确保间隔
    t_d3 = t_d2 + x(5); % 确保间隔
    del1 = x(6); del2 = x(7); del3 = x(8);
    
    % 2. 计算3枚弹的运动学参数
    [init_pos_m1, v_vec_m1, target_pos] = get_env_params(v, theta);
    
    [exp1, pos1] = get_bomb_kinematics(v, theta, t_d1, del1);
    [exp2, pos2] = get_bomb_kinematics(v, theta, t_d2, del2);
    [exp3, pos3] = get_bomb_kinematics(v, theta, t_d3, del3);
    
    % 物理约束:任何一枚在地下爆炸都给予惩罚
    if pos1(3)<0 || pos2(3)<0 || pos3(3)<0
        fitness = 1e6; return;
    end
    
    % 3. 确定仿真时间范围
    smoke_dur = 20;
    t_hit = norm(init_pos_m1) / 300;
    
    % 整个遮蔽过程是3个区间的并集,我们取最早起爆到最晚结束
    t_start_global = min([exp1, exp2, exp3]);
    t_end_global = min(max([exp1, exp2, exp3]) + smoke_dur, t_hit);
    
    if t_start_global >= t_end_global
        fitness = 1e5; return;
    end
    
    % --- 智能扫描策略 ---
    
    % A. 粗扫描 (0.1s)
    dt_coarse = 0.1;
    T_coarse = t_start_global : dt_coarse : t_end_global;
    if isempty(T_coarse), fitness = 1e5; return; end
    
    min_dist_global = inf;
    potential_times = [];
    
    % 预计算导弹位置
    Pm_c = init_pos_m1 + v_vec_m1 .* T_coarse(:);
    Pt = target_pos;
    
    for k = 1:length(T_coarse)
        t = T_coarse(k);
        pm = Pm_c(k,:);
        vec_line = Pt - pm;
        norm_line = norm(vec_line);
        
        dist_min_k = inf;
        
        % 检查3个烟雾团
        clouds = [exp1, pos1; exp2, pos2; exp3, pos3]; % 3x4 matrix
        for m = 1:3
            t_exp = clouds(m, 1);
            p_exp = clouds(m, 2:4);
            
            % 只有在烟雾存续期内才计算
            if t >= t_exp && t <= t_exp + smoke_dur
                ps = p_exp;
                ps(3) = ps(3) - 3 * (t - t_exp); % 下沉
                
                vec_point = ps - pm;
                d = norm(cross(vec_point, vec_line)) / norm_line;
                if d < dist_min_k, dist_min_k = d; end
            end
        end
        
        if dist_min_k < min_dist_global, min_dist_global = dist_min_k; end
        if dist_min_k < 18, potential_times = [potential_times; t]; end
    end
    
    % B. 精细计算
    if isempty(potential_times)
        fitness = min_dist_global;
    else
        dt_fine = 0.0001;
        window = 0.06;
        all_fine_times = [];
        
        for t_center = potential_times'
            t_sub = (max(t_start_global, t_center - window) : dt_fine : min(t_end_global, t_center + window))';
            t_sub = round(t_sub / dt_fine) * dt_fine; % 网格对齐
            all_fine_times = [all_fine_times; t_sub];
        end
        
        if isempty(all_fine_times), fitness = min_dist_global; return; end
        T_fine = unique(all_fine_times);
        
        Pm_f = init_pos_m1 + v_vec_m1 .* T_fine;
        
        mask_count = 0;
        
        % 向量化比较麻烦,这里用循环处理这3个球
        % 为了加速,把3个球的信息展开
        exp_times = [exp1, exp2, exp3];
        pos_starts = [pos1; pos2; pos3];
        
        for j = 1:length(T_fine)
            t = T_fine(j);
            pm = Pm_f(j,:);
            vec_line = Pt - pm;
            norm_line = norm(vec_line);
            
            is_masked = false;
            
            % 遍历3个球
            for m = 1:3
                if t >= exp_times(m) && t <= exp_times(m) + smoke_dur
                    ps = pos_starts(m, :);
                    ps(3) = ps(3) - 3 * (t - exp_times(m));
                    
                    vec_point = ps - pm;
                    % 快速判断距离平方
                    cp = cross(vec_point, vec_line);
                    d2 = sum(cp.^2) / (norm_line^2);
                    
                    if d2 <= 100 % 10^2
                        is_masked = true;
                        break; % 只要被任意一个遮住就算遮住
                    end
                end
            end
            
            if is_masked
                mask_count = mask_count + 1;
            end
        end
        
        fitness = -(mask_count * dt_fine) - 10000;
    end
end

function mask_time = calculate_masking_pure_multi(x, dt)
    % 解码
    v = x(1); theta = x(2);
    t_d1 = x(3); t_d2 = t_d1 + x(4); t_d3 = t_d2 + x(5);
    del1 = x(6); del2 = x(7); del3 = x(8);
    
    [init_pos_m1, v_vec_m1, target_pos] = get_env_params(v, theta);
    [exp1, pos1] = get_bomb_kinematics(v, theta, t_d1, del1);
    [exp2, pos2] = get_bomb_kinematics(v, theta, t_d2, del2);
    [exp3, pos3] = get_bomb_kinematics(v, theta, t_d3, del3);
    
    smoke_dur = 20;
    t_hit = norm(init_pos_m1) / 300;
    t_start = min([exp1, exp2, exp3]);
    t_end = min(max([exp1, exp2, exp3]) + smoke_dur, t_hit);
    
    if t_start >= t_end, mask_time=0; return; end
    
    T = t_start : dt : t_end;
    count = 0;
    
    exp_times = [exp1, exp2, exp3];
    pos_starts = [pos1; pos2; pos3];
    
    for k = 1:length(T)
        t = T(k);
        pm = init_pos_m1 + v_vec_m1 * t;
        vec_line = target_pos - pm;
        norm_line = norm(vec_line);
        
        masked = false;
        for m = 1:3
            if t >= exp_times(m) && t <= exp_times(m) + smoke_dur
                ps = pos_starts(m, :);
                ps(3) = ps(3) - 3 * (t - exp_times(m));
                
                vec_point = ps - pm;
                d = norm(cross(vec_point, vec_line)) / norm_line;
                if d <= 10
                    masked = true; break;
                end
            end
        end
        if masked, count = count + 1; end
    end
    mask_time = count * dt;
end

function [init_pos_m1, v_vec_m1, target_pos] = get_env_params(v, theta)
    init_pos_m1 = [20000, 0, 2000];
    fake_target = [0, 0, 0];
    dir_m1 = (fake_target - init_pos_m1) / norm(fake_target - init_pos_m1);
    v_vec_m1 = dir_m1 * 300;
    target_pos = [0, 200, 0];
end

function [t_explode, explode_pos] = get_bomb_kinematics(v_mag, theta, t_drop, t_delay)
    init_pos_uav = [17800, 0, 1800];
    v_vec_uav = [v_mag * cos(theta), v_mag * sin(theta), 0];
    drop_pos = init_pos_uav + v_vec_uav * t_drop;
    
    g = 9.8;
    explode_pos = drop_pos;
    explode_pos(1) = drop_pos(1) + v_vec_uav(1) * t_delay;
    explode_pos(2) = drop_pos(2) + v_vec_uav(2) * t_delay;
    explode_pos(3) = drop_pos(3) - 0.5 * g * t_delay^2;
    
    t_explode = t_drop + t_delay;
end
\end{lstlisting}
\begin{lstlisting}[language=Matlab,caption={Problem 4}]
function Problem4_MultiUAV_Strict()
    clc; close all;

    %% 1. 全局环境参数 (Env)
    Env.g = 9.8; 
    Env.Pos_True = [0, 200, 0]; 
    Env.Pos_M1   = [20000, 0, 2000];       
    Env.V_M1     = 300;                           
    Env.Vec_M    = [0, 0, 0] - Env.Pos_M1; 
    Env.Dist_M1  = norm(Env.Vec_M);
    Env.Dir_M1   = Env.Vec_M / Env.Dist_M1;         
    Env.V_sink   = 3;      
    Env.R_smk    = 10;           
    Env.T_last   = 20; 
    
    Env.Pos_UAVs = [
        17800, 0,    1800;  % FY1
        12000, 1400, 1400;  % FY2
        6000, -3000, 700    % FY3
    ];

    %% 2. 优化参数设置 (已修正速度约束)
    
    % 估算基准航向
    Base_Ang1 = rad2deg(atan2(200 - 0,    0 - 17800)); 
    Base_Ang2 = rad2deg(atan2(200 - 1400, 0 - 12000)); 
    Base_Ang3 = rad2deg(atan2(200 + 3000, 0 - 6000));  
    
    % === [修正点]:将速度下界从 60 改为 70 ===
    % 变量顺序: [V, Ang, T_drop, T_delay] * 3架
    LB = [70, Base_Ang1-45, 0, 1,   70, Base_Ang2-45, 0, 1,   70, Base_Ang3-45, 0, 1];
    UB = [140, Base_Ang1+45, 15, 8, 140, Base_Ang2+45, 35, 8, 140, Base_Ang3+45, 55, 8];
    
    de_opts.NP = 200;       
    de_opts.MaxIter = 800;  
    de_opts.F = 0.5;        
    de_opts.CR = 0.9;       

    fprintf('======================================================\n');
    fprintf('      问题4:三机协同立体封锁 (速度修正版 70-140)\n');
    fprintf('======================================================\n');
    
    try, if isempty(gcp('nocreate')), parpool; end; end

    CostFunc = @(x) CostFunc_3UAV(x, Env);

    tic;
    [best_x, best_val, curve] = Run_DE_3UAV(CostFunc, LB, UB, de_opts);
    total_time = toc;

    %% 3. 结果解析与输出
    [~, real_shield_time] = CostFunc_3UAV(best_x, Env);
    Res = Parse_Params(best_x);

    fprintf('\n>>> 优化完成!耗时: %.2f 秒\n', total_time);
    fprintf('>>> 🏆 最终最大遮蔽时长: %.4f 秒\n', real_shield_time);
    
    fprintf('\n[战术指令表]\n');
    fprintf('| 机号 | 初始位置(X,Y,Z) | 速度(m/s) | 航向(deg) | 投放(s) | 延时(s) | 起爆(s) |\n');
    fprintf('|------|-----------------|-----------|-----------|---------|---------|---------|\n');
    fprintf('| FY1  | (%5d,%5d,%4d) | %9.2f | %9.2f | %7.2f | %7.2f | %7.2f |\n', ...
        Env.Pos_UAVs(1,:), Res(1).v, Res(1).a, Res(1).td, Res(1).ty, Res(1).tp);
    fprintf('| FY2  | (%5d,%5d,%4d) | %9.2f | %9.2f | %7.2f | %7.2f | %7.2f |\n', ...
        Env.Pos_UAVs(2,:), Res(2).v, Res(2).a, Res(2).td, Res(2).ty, Res(2).tp);
    fprintf('| FY3  | (%5d,%5d,%4d) | %9.2f | %9.2f | %7.2f | %7.2f | %7.2f |\n', ...
        Env.Pos_UAVs(3,:), Res(3).v, Res(3).a, Res(3).td, Res(3).ty, Res(3).tp);

    %% 4. 绘图
    Plot_3UAV_Scenario(Res, real_shield_time, curve, Env);
    Save_To_Excel(Res, 'result2.xlsx');
end

%% =================================================================
%  局部函数库 (保持不变,仅 CostFunc_3UAV 内部也加一道保险)
% =================================================================

function Res = Parse_Params(x)
    for i = 1:3
        idx = (i-1)*4;
        Res(i).v  = x(idx+1);
        Res(i).a  = x(idx+2);
        Res(i).td = x(idx+3);
        Res(i).ty = x(idx+4);
        Res(i).tp = Res(i).td + Res(i).ty;
    end
end

function [score, pure_time] = CostFunc_3UAV(x, Env)
    Res = Parse_Params(x);
    
    % --- [保险修正]:并在内部逻辑中再次检查边界 ---
    for i=1:3
        if Res(i).v < 70 || Res(i).v > 140 % 严格限制 70-140
            score = 1e6; pure_time = 0; return; 
        end
    end
    
    % ... (其余物理计算逻辑与之前完全一致,此处省略以节省篇幅) ...
    % 您之前复制的 CostFunc_3UAV 代码逻辑在这里保持不变
    
    % ---------------- 粘贴之前的 CostFunc_3UAV 逻辑 ----------------
    P_Pop = zeros(3, 3);
    Times_Pop = zeros(3, 1);
    for i = 1:3
        ang_rad = deg2rad(Res(i).a);
        Vel_Vec = [cos(ang_rad), sin(ang_rad), 0] * Res(i).v;
        P_Drop = Env.Pos_UAVs(i,:) + Vel_Vec * Res(i).td;
        Disp = Vel_Vec * Res(i).ty;
        Disp(3) = Disp(3) - 0.5 * 9.8 * Res(i).ty^2;
        P_Pop(i,:) = P_Drop + Disp;
        Times_Pop(i) = Res(i).tp;
        if P_Pop(i,3) < 0, score = 1e5; pure_time = 0; return; end
    end
    
    t_start = min(Times_Pop);
    t_end = min(Env.Dist_M1/Env.V_M1, max(Times_Pop) + Env.T_last);
    if t_start >= t_end, score = 1e5; pure_time = 0; return; end
    
    Target_Points = [0, 200, 5; 0, 200, 10; 0, 200, 0; -7, 200, 5; 7, 200, 5];
    num_pts = 5;
    dt = 0.05; t_vec = t_start : dt : t_end;
    total_coverage = 0;
    penalty_dist = 0;
    min_dists = [1e9, 1e9, 1e9]; 
    
    for t = t_vec
        P_M = Env.Pos_M1 + Env.Dir_M1 * (Env.V_M1 * t);
        blocked_pts_count = 0;
        for p = 1:num_pts
            TP = Target_Points(p,:);
            v_los = TP - P_M;
            len_los_sq = sum(v_los.^2);
            is_pt_blocked = false;
            for k = 1:3
                if t >= Times_Pop(k) && t <= Times_Pop(k) + Env.T_last
                    P_Smk = P_Pop(k,:) - [0, 0, Env.V_sink * (t - Times_Pop(k))];
                    w_vec = P_Smk - P_M;
                    c1 = dot(w_vec, v_los);
                    if c1 > 0
                        b = max(0, min(1, c1 / len_los_sq));
                        Pb = P_M + v_los * b;
                        d_sq = sum((P_Smk - Pb).^2);
                        dist = sqrt(d_sq);
                        if p == 1 && dist < min_dists(k), min_dists(k) = dist; end
                        if d_sq <= Env.R_smk^2, is_pt_blocked = true; end
                    end
                end
            end
            if is_pt_blocked, blocked_pts_count = blocked_pts_count + 1; end
        end
        total_coverage = total_coverage + (blocked_pts_count / num_pts) * dt;
    end
    pure_time = total_coverage;
    for k=1:3, penalty_dist = penalty_dist + max(0, min_dists(k) - Env.R_smk); end
    score = -total_coverage + 0.1 * penalty_dist;
end

%% --- DE 求解器 (修正种子速度) ---
function [best_mem, best_val, curve] = Run_DE_3UAV(cost_func, lb, ub, opts)
    NP = opts.NP; D = length(lb);
    pop = repmat(lb, NP, 1) + rand(NP, D) .* repmat(ub-lb, NP, 1);
    
    % === [修正点]:种子的速度也必须符合 70-140 约束 ===
    % FY1 修正为 73 (低速流最优), FY2/FY3 为 120 (高速流)
    pop(1,:) = [73, 180, 0, 3,   120, 190, 15, 4,   120, 150, 35, 5];
    pop(2,:) = [140, 180, 0, 3,  140, 190, 20, 3,   140, 150, 40, 3];
    
    val = zeros(NP, 1);
    parfor i=1:NP, val(i) = cost_func(pop(i,:)); end
    [best_val, idx] = min(val);
    best_mem = pop(idx, :);
    curve = zeros(opts.MaxIter, 1);
    
    h = waitbar(0, 'Multi-UAV Optimization...');
    for gen = 1 : opts.MaxIter
        F = opts.F * (1 - 0.3 * gen/opts.MaxIter);
        pop_new = pop; val_new = val;
        parfor i = 1 : NP
            r = randperm(NP, 3);
            mutant = best_mem + F * (pop(r(1),:) - pop(r(2),:));
            trial = pop(i, :);
            j_rand = randi(D);
            for j = 1 : D
                if rand < opts.CR || j == j_rand, trial(j) = mutant(j); end
            end
            trial = max(trial, lb); trial = min(trial, ub);
            t_v = feval(cost_func, trial);
            if t_v < val(i), pop_new(i,:) = trial; val_new(i) = t_v; end
        end
        pop = pop_new; val = val_new;
        [c_best, idx] = min(val);
        if c_best < best_val, best_val = c_best; best_mem = pop(idx, :); end
        curve(gen) = -best_val;
        if mod(gen, 50) == 0
            [~, t_real] = feval(cost_func, best_mem);
            waitbar(gen/opts.MaxIter, h, sprintf('Iter %d: %.2fs', gen, t_real));
            fprintf('    Iter %d: 引导得分 %.4f | 真实遮蔽 %.4fs\n', gen, best_val, t_real);
        end
    end
    close(h);
end

%% --- 绘图与保存 (保持不变) ---
function Plot_3UAV_Scenario(Res, score, curve, Env)
    figure('Color','w', 'Position', [50, 50, 1200, 600]);
    subplot(1, 2, 1); hold on; grid on; axis equal; view([-30, 30]);
    plot3(Env.Pos_True(1), Env.Pos_True(2), Env.Pos_True(3), 'go', 'MarkerSize', 12, 'MarkerFaceColor','g');
    plot3(Env.Pos_M1(1), Env.Pos_M1(2), Env.Pos_M1(3), 'rv', 'MarkerSize', 10, 'MarkerFaceColor','r');
    colors = {'b', 'm', [0 0.5 0]};
    for i = 1:3
        ang = deg2rad(Res(i).a);
        V = [cos(ang), sin(ang), 0] * Res(i).v;
        P0 = Env.Pos_UAVs(i,:);
        P_Drop = P0 + V * Res(i).td;
        plot3([P0(1) P_Drop(1)], [P0(2) P_Drop(2)], [P0(3) P_Drop(3)], '-', 'Color', colors{i}, 'LineWidth', 1.5);
        text(P0(1), P0(2), P0(3)+200, sprintf('FY%d',i), 'Color', colors{i});
        P_Pop = P_Drop + V * Res(i).ty - [0,0,0.5*9.8*Res(i).ty^2];
        plot3(P_Pop(1), P_Pop(2), P_Pop(3), 'h', 'MarkerSize', 14, 'MarkerFaceColor', colors{i});
        plot3([P_Drop(1) P_Pop(1)], [P_Drop(2) P_Pop(2)], [P_Drop(3) P_Pop(3)], 'k:');
    end
    line([Env.Pos_M1(1) Env.Pos_True(1)], [Env.Pos_M1(2) Env.Pos_True(2)], [Env.Pos_M1(3) Env.Pos_True(3)], ...
        'Color', [0.6 0.6 0.6], 'LineStyle', '--');
    xlabel('X'); ylabel('Y'); zlabel('Z');
    title('三机协同立体封锁示意图');
    subplot(1, 2, 2); hold on; grid on;
    for i=1:3
        tp = Res(i).tp;
        fill([tp, tp+Env.T_last, tp+Env.T_last, tp], [i-0.3, i-0.3, i+0.3, i+0.3], colors{i}, 'FaceAlpha', 0.6);
        text(tp, i, sprintf(' Start: %.1fs', tp));
    end
    title(['协同遮蔽时序 (总长: ' num2str(score) 's)']);
    xlabel('时间 (s)'); ylim([0 4]); yticks([1 2 3]); yticklabels({'FY1','FY2','FY3'});
end

function Save_To_Excel(Res, filename)
    data = zeros(3, 5);
    for i=1:3
        data(i,1) = Res(i).v;
        data(i,2) = Res(i).a;
        data(i,3) = Res(i).td;
        data(i,4) = Res(i).ty;
        data(i,5) = Res(i).tp;
    end
    try, writematrix(data, filename); catch, end
end
\end{lstlisting}
\end{document}